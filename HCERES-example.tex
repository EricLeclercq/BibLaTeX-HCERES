\documentclass[10pt]{article}
\usepackage{a4}
\usepackage[french]{babel}

\usepackage[backend=biber,sorting=ydnt,defernumbers=true]{biblatex}


%\addbibresource{./Biblio/conf-int.bib}

\usepackage{filecontents}
\begin{filecontents}{references.bib}
@book{knuth1986texbook,
  title={The texbook},
  author={Knuth, D.E. and Bibby, D.},
  volume={1993},
  year={1986},
  publisher={Addison-Wesley},
  keywords={OS}
}
@article{knuth1977fast,
  title={Fast pattern matching in strings},
  author={Knuth, D.E. and Morris Jr, J.H. and Pratt, V.R.},
  journal={SIAM journal on computing},
  volume={6},
  number={2},
  pages={323--350},
  year={1977},
  publisher={SIAM},
  keywords={ACL}
}
@inproceedings{knuth1970simple,
  title={Simple word problems in universal algebras},
  author={Knuth, D.E. and Bendix, P.B.},
  booktitle={Computational problems in abstract algebra},
  volume={263},
  pages={297},
  year={1970},
  keywords={ACL}
}
\end{filecontents}
\addbibresource{references.bib}

% ne pas oublier d'utiliser biber à la place de bibtex
\begin{document}
\begin{center}
\Large
\textsf{\bfseries Exemple de bibliographies multiples avec codage HCERES}
\end{center}

\begin{itemize}
\item  Publications scientifiques
 \begin{itemize}
    \item ACL : Articles dans des revues internationales ou nationales avec comité de lecture répertoriées par l'HCERES ou dans les bases de données internationales (ISI Web of Knowledge, Pub Med, Scopus...).
    \item ACLN : Articles dans des revues avec comité de lecture non répertoriées par l'HCERES ou dans des bases de données internationales.
    \item ASCL : Articles dans des revues sans comité de lecture.
    \item OS : Ouvrages scientifiques (y compris les éditions critiques et les traductions scientifiques)
    \item PT : Publications de transfert.
    \item BRE : Brevets
    \item C-INV : Conférences données à l'invitation du Comité d'organisation dans un congrès national ou international.
    \item C-ACTI : Communications avec actes dans un congrès international.
    \item C-ACTN : Communications avec actes dans un congrès national.
    \item C-COM : Communications orales sans actes dans un congrès international ou national.
    \item C-AFF : Communications par affiche dans un congrès international ou national.
    \item DO : Directions d'ouvrages ou de revues.
    \item OR : Outils de recherche (bases de données, corpus de recherche...).
 \end{itemize}
 \item Diffusions de la culture scientifique
  \begin{itemize}
    \item PV : Publications de vulgarisation.
    \item PAT : Productions artistiques théorisées (compositions musicales, cinématographiques, expositions, installations...).
  \end{itemize}
 \item Autres productions
 \begin{itemize}
   \item  AP : Autres productions. Bases de données, logiciels enregistrés, comptes rendus d'ouvrages, rapports de fouilles, guides techniques, catalogues d'exposition, rapports intermédiaires de grands projets internationaux, etc.
   \item TH : Thèses de doctorat 
 \end{itemize}
\end{itemize}
  

\nocite{*}

\newrefcontext[labelprefix=OS]
\printbibliography[title={Ouvrages, chapitres de livres},keyword=OS]

\newrefcontext[labelprefix=ACL]
\printbibliography[title={Revues},keyword=ACL]
\end{document}